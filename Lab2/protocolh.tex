 \documentclass{article}
 
 \usepackage{fancyhdr}
 \usepackage{cmap}
 \usepackage[T2A]{fontenc}
 \usepackage[utf8]{inputenc}
 \usepackage[english,russian,ukrainian]{babel}
 \usepackage{pythontex}
 \usepackage{indentfirst}
 \usepackage{graphics}
 \usepackage{hyperref}
 \hypersetup{
 	colorlinks,
 	citecolor=black,
 	linkcolor=black,
 	filecolor=black,
 	urlcolor=black
 }
 
 \usepackage{geometry}
 \geometry{
 	a4paper,
 	total={155mm,247mm},
 	left=30mm,
 	top=30mm,
 }
 \usepackage{titlesec}
 \usepackage{fancyhdr}
 
 \pagestyle{fancy}
 \fancyhf{}
 \renewcommand{\headrulewidth}{0.5pt}
 \renewcommand{\footrulewidth}{0.5pt}
 \rhead{\LaTeX}
 \lhead{Вячеслав Козачок}
 
 %\titleformat{\section}
 %{\normalfont\Large\bfseries}{\thesection}{1em}{}
 
 %\titleformat*{\subsection}{\large\bfseries}
 
 \begin{document}
 	
	\begin{titlepage}
	
	\begin{frame}[t]
		\raisebox{-10mm}[10pt][0pt]{%
			\makebox[\textwidth][c]{\includegraphics[width=\textwidth]{institute.jpg}}
		}
	\end{frame}\\
	\large
	\centering{
		\vspace{5mm}\\
		Міністерство освіти і науки України \\
		Національний технічний університет України \\
		"Київський політехнічний інститут імені Ігоря Сікорського"\\
		Фізико-технічний інститут
	}
	\vspace{3cm}\\
	\centering{
		{\Huge\textbf{Операційні системи} \\ \vspace{0.15cm}}
		{\huge Лабораторна №4}
	}
	
	\vspace{8cm}
	\Large
	\begin{flushright}
		Виконав:\\
		Студент групи ФБ-82\\
		\textbf{Козачок Вячеслав}\\
		Перевірив: \\
		Кіреєнко О.В.
	\end{flushright}
	\vfill
	
	\centering{Київ - 2020}
	
\end{titlepage}
	\newpage
	\Large
	\section*{Завдання до виконання}
	\large
	\begin{enumerate}
	\item Створіть каталог \verb|lab_2.|
	\item Скопіюйте в каталог \verb|lab_2| файл \verb|/bin/cat| під назвою \verb|my_cat|.
	\item За допомогою файлу \verb|my_cat|, що знаходиться в каталозі \verb|lab_2|,
	перегляньте уміст файлу \verb|.profile| (ви знаходитесь у домашньому
	каталозі).
	\item Перегляньте список файлів у каталозі \verb|lab_2|. Потім перегляньте
	список усіх файлів, включаючи приховані, з повною інформацією про
	файли. Зверніть увагу на права доступу, власника, дату модифікації
	файлу, що ви тільки-но скопіювали. Потім перегляньте цю інформацію
	про оригінальний файл (той, який копіювали) і порівняйте два
	результати.
	\item Змініть права доступу до файлу \verb|my_cat| так, щоб власник міг тільки
	читати цей файл.
	\item Переконайтеся в тому, що ви зробили ці зміни і повторіть п.3.
	\item Визначте права на файл \verb|my_cat| таким чином, щоб ви могли робити з
	файлом усе, що завгодно, а всі інші — нічого не могли робити.
	\item Поверніться в домашній каталог. Змініть права доступу до каталогу
	\verb|lab_2| так, щоб ви могли його тільки читати.
	\item Спробуйте переглянути простий список файлів у цьому каталозі.
	Спробуйте переглянути список файлів з повною інформацією про них.
	Спробуйте запустити і видалити файл \verb|my_cat| з цього каталогу.
	\item Поясніть отримані результати. Результати виконання п.8 можуть бути
	різними в різних версіях UNIX, зокрема, Linux і FreeBSD.
	Прокоментуйте отримані результати у висновках.
	\item За допомогою команди \verb|su <user name>|, завантажтесь в систему,
	користуючись обліковим записом іншого користувача. (Вам потрібно
	знати пароль цього користувача.) Спробуйте отримати доступ до
	Вашого каталогу \verb|lab_2| . Перевірте, чи правильно зроблено завдання
	попереднього пункту. Створіть каталог \verb|lab_2_2|.
	\item Знову завантажтесь в систему, користуючись своїм обліковим
	записом. Спробуйте зробити власником каталогу \verb|lab_2| іншого
	користувача. Спробуйте зробити себе власником каталогу \verb|lab_2_2|.
	Поясніть результати.	
	\item . Зайдіть у каталог \verb|lab_2|. Зробіть так, щоб нові створені файли і
	каталоги діставали права доступу згідно Таблиці. Створіть новий файл
	і каталог і переконайтеся в правильності ваших установок.
	
	\textbf{Варіант 9:} Файли 644, каталоги 744.
	 
	 \item Поверніть собі права читати, писати, та переглядати вміст каталогів.
	 \item Створіть у каталозі \verb|lab_2| каталог \verb|acl_test| та у ньому файли
	 \verb|file1, file2.profile| Після час створення \verb|file1| додайте у нього
	 довільний текст.
	 \item Виведіть \verb|ACL| для \verb|file1|
	 \item Змінить права доступу на \verb|file1| так, щоб тільки власник мав право
	 на читання.
	 \item Увійдіть до системи під іншим обліковим записом та спробуйте
	 прочитати вміст \verb|file1|. Що отримаємо? Поверніться до свого
	 облікового запису.
	 \item За допомогою команди \verb|setfacl| додайте право на читання іншому
	 обраному користувачу для \verb|file1| Перевірте, що створився новий
	 ACL для \verb|file1|
	 \item Увійдіть до системи під іншим обліковим записом та спробуйте
	 прочитати вміст \verb|file1|. Що отримаємо? Поверніться до свого
	 облікового запису.
	 \item За допомогою команди \verb|setfacl| встановіть значення маски таким
	 чином щоб дозволити читати вміст \verb|file1| іншому користувачу.
	 Виведіть ACL для \verb|file1|
	 \item Увійдіть до системи під іншим обліковим записом, та спробуйте
	 прочитати вміст \verb|file1|. Ви повинні мати таку змогу.
	 
	\newpage
	
	\Large
\section*{Хід роботи}
	
\end{enumerate}
	
\large
\begin{enumerate}
	\item \begin{verbatim}
	[eski@eski-pc Lab2]$ mkdir lab2
	\end{verbatim}
	\item \begin{verbatim}
	[eski@eski-pc lab_2]$ cp /bin/cat my_cat
	\end{verbatim}
	\item \begin{verbatim}
	[eski@eski-pc lab_2]$ ./my_cat ~/.profile 
	export EDITOR=/usr/bin/nano
	alias cduni="cd /home/eski/Documents/University/2\ курс/"
	\end{verbatim}
	\item \begin{verbatim}
	[eski@eski-pc lab_2]$ ls
	my_cat
	
	[eski@eski-pc lab_2]$ ls -la
	total 48
	drwxr-xr-x 2 eski eski  4096 Feb 20 15:01 .
	drwxr-xr-x 3 eski eski  4096 Feb 20 16:19 ..
	-rwxr-xr-x 1 eski eski 39048 Feb 20 15:01 my_cat
	
	[eski@eski-pc lab_2]$ ls -la /usr/bin
	.............
	lrwxrwxrwx  1 root   root           5 Nov 12 14:52 bzcat -> bzip2
	-rwxr-xr-x  1 root   root       39048 Nov 12 13:00 cat
	-rwxr-xr-x  1 root   root        3272 Nov 13 18:16 catchsegv
	.............
	\end{verbatim}
	\item \begin{verbatim}
	[eski@eski-pc lab_2]$ chmod u=r-- my_cat

	
	\end{verbatim}
	\item \begin{verbatim}
	[eski@eski-pc lab_2]$ ls -l
	total 40
	-r--r-xr-x 1 eski eski 39048 Feb 20 15:01 my_cat
	
	[eski@eski-pc lab_2]$ ./my_cat ~/.profile
	bash: ./my_cat: Permission denied
	
	\end{verbatim}
	\item \begin{verbatim}[eski@eski-pc lab_2]$ chmod u=rwx,g=---,o=--- my_cat 
	[eski@eski-pc lab_2]$ ls -l
	total 40
	-rwx------ 1 eski eski 39048 Feb 20 15:01 my_cat
	
	\end{verbatim}
	\item \begin{verbatim}
	[eski@eski-pc lab_2]$ cd ..
	[eski@eski-pc Lab2]$ chmod u=r-- lab_2/
	[eski@eski-pc Lab2]$ ls -l
	total 256
	dr--r-xr-x 2 eski eski   4096 Feb 20 17:06 lab_2
	
	\end{verbatim}
	\newpage
	\item \begin{verbatim}
	[eski@eski-pc Lab2]$ ls lab_2
	ls: cannot access 'lab_2/my_cat': Permission denied
	my_cat
	
	[eski@eski-pc Lab2]$ ls -l lab_2
	ls: cannot access 'lab_2/my_cat': Permission denied
	total 0
	-????????? ? ? ? ?            ? my_cat
	
	[eski@eski-pc Lab2]$ ./lab_2/my_cat ~/.profile 
	bash: ./lab_2/my_cat: Permission denied
	
	[eski@eski-pc Lab2]$ rm lab_2/my_cat 
	rm: cannot remove 'lab_2/my_cat': Permission denied
	
	\end{verbatim}
	\item Дивіться висновок(*)
	\item \begin{verbatim}
	[eski@eski-pc Lab2]$ su topshop
	Password: 
	
	[topshop@eski-pc Lab2]$ cd lab_2/
	
	[topshop@eski-pc lab_2]$ mkdir lab_2_2
	mkdir: cannot create directory ‘lab_2_2’: Permission denied
	# Треба додати до lab_2 правно на запис іншим користувачам
	
	[topshop@eski-pc lab_2]$ exit
	
	[eski@eski-pc Lab2]$ ls -l 
	total 264
	dr--r-xr-x 2 eski eski   4096 Feb 20 17:06 lab_2
	[eski@eski-pc Lab2]$ chmod o+w lab_2/
	
	[eski@eski-pc Lab2]$ su topshop
	Password: 
	[topshop@eski-pc Lab2]$ cd lab_2/
	[topshop@eski-pc lab_2]$ mkdir lab_2_2
	\end{verbatim}
	\item \begin{verbatim}
	[eski@eski-pc Lab2]$ chown topshop lab_2
	chown: changing ownership of 'lab_2': Operation not permitted
	
	[eski@eski-pc Lab2]$ chown topshop lab_2/lab_2_2 
	chown: cannot access 'lab_2/lab_2_2': Permission denied
	\end{verbatim}
	\newpage
	\item \begin{verbatim}
	[eski@eski-pc lab_2]$ umask 0033
	
	[eski@eski-pc lab_2]$ ls -l
	total 48
	drwxr--r-- 2 eski    eski     4096 Feb 20 22:53 dir
	-rw-r--r-- 1 eski    eski        0 Feb 20 22:53 file
	\end{verbatim}
	\item \begin{verbatim}
	[eski@eski-pc Lab2]$ cd lab_2/
	[eski@eski-pc lab_2]$ chmod u+xw .
	\end{verbatim}
	\item \begin{verbatim}
	[eski@eski-pc lab_2]$ mkdir acl_test
	[eski@eski-pc lab_2]$ touch acl_test/file1
	[eski@eski-pc lab_2]$ touch acl_test/file2
	[eski@eski-pc lab_2]$ echo Some random text >> acl_test/file1

	\end{verbatim}
	\item \begin{verbatim}
	[eski@eski-pc lab_2]$ cd acl_test/
	[eski@eski-pc acl_test]$ getfacl file1
	# file: file1
	# owner: eski
	# group: eski
	user::rw-
	group::r--
	other::r--
	
	\end{verbatim}
	\item \begin{verbatim}
	setfacl -m m::wx file1
	\end{verbatim}
	\item \begin{verbatim}
	[eski@eski-pc acl_test]$ su topshop 
	Password: 
	[topshop@eski-pc acl_test]$ cat file1
	cat: file1: Permission denied
	[topshop@eski-pc acl_test]$ exit
	
	\end{verbatim}
	\item \begin{verbatim}
	[eski@eski-pc acl_test]$ setfacl -m u:topshop:r  file1
	[eski@eski-pc acl_test]$ getfacl file1
	# file: file1
	# owner: eski
	# group: eski
	user::rw-
	user:topshop:r--
	group::r--
	mask::r--
	other::r--
	
	\end{verbatim}
	\item \begin{verbatim}
	[eski@eski-pc acl_test]$ su topshop
	Password: 
	[topshop@eski-pc acl_test]$ cat file1
	cat: file1: Permission denied
	[topshop@eski-pc acl_test]$ exit 
	\end{verbatim}
	\item \begin{verbatim}
	[eski@eski-pc acl_test]$ setfacl -m m:r file1
	[eski@eski-pc acl_test]$ getfacl file1
	# file: file1
	# owner: eski
	# group: eski
	user::rw-
	user:topshop:r--
	group::r--
	mask::r--
	other::r--
	
	
	\end{verbatim}
	\item \begin{verbatim}
	[eski@eski-pc acl_test]$ su topshop 
	Password: 
	[topshop@eski-pc acl_test]$ cat file1
	Some random text la la la
	
	\end{verbatim}
		
\end{enumerate}
	
\newpage
\Large
\section*{Висновок}
\large
\textbf{Відповідь на запитання до завдання:} Для каталогів право виконання трактується як право доступу до таблиці індексних дескрипторів на читання і запис, не маючи цього права
неможливо зробити поточним цей каталог чи будь-який з його підкаталогів,
неможливо ознайомитись і змінити права доступу до об’єктів цього каталогу,
можна тільки переглядати його вміст, якщо є право читання. Навіть маючи
право запису, без права виконання не можна змінити вміст каталогу.
Навпаки, якщо є право на виконання, але не встановлено право на читання
для каталогу, то неможливо переглянути вміст каталогу, але можна заходити
в його підкаталоги чи звертатись до файлів, що містяться в ньому, якщо
знати їхні імена.
\\\\
\indent Таким чином в лабораторній роботі ми навчились задавати права на каталоги та директорії різними способами. 
\begin{itemize}
	\item chmod
	\item setfacl
\end{itemize}
Зрозуміли значення прав для директорій, файлів. Призначення групи власника, самого власника, призначення прав для інших користувачів.	
 \end{document}




