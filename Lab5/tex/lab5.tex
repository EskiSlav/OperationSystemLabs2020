\documentclass{article}

\usepackage{fancyhdr}
\usepackage{cmap}
\usepackage[T2A]{fontenc}
\usepackage[utf8]{inputenc}
\usepackage[english,russian,ukrainian]{babel}
\usepackage{indentfirst}
\usepackage{graphicx}
\usepackage{hyperref}
\usepackage{titlesec}
\usepackage{fancyhdr}
\usepackage{geometry}
\usepackage{scrextend} % For margin
\hypersetup{
	colorlinks,
	citecolor=black,
	linkcolor=black,
	filecolor=black,
	urlcolor=black
}

\geometry{
	a4paper,
	total={165mm,247mm},
	left=20mm,
	top=30mm,
}

\pagestyle{fancy}
\fancyhf{}
\renewcommand{\headrulewidth}{0.5pt}
%\renewcommand{\footrulewidth}{0.1pt}
\rhead{\LaTeX}
\lhead{Вячеслав Козачок}

\cfoot{\thepage}
%\titleformat{\section}
%{\normalfont\Large\bfseries}{\thesection}{1em}{}

%\titleformat*{\subsection}{\large\bfseries}

 \usepackage{listings}
 \usepackage{xcolor}
 
 \definecolor{codegreen}{rgb}{0,0.6,0}
 \definecolor{codegray}{rgb}{0.5,0.5,0.5}
 \definecolor{codepurple}{rgb}{0.58,0,0.82}
 \definecolor{backcolour}{rgb}{0.99,0.99,0.99}
 \definecolor{codeblue}{rgb}{0.1,0.1,0.99}
 
 \lstdefinestyle{mystyle}{
 	backgroundcolor=\color{backcolour},   
 	commentstyle=\color{codegreen},
 	keywordstyle=\color{codeblue},
 	numberstyle=\scriptsize\color{codegray},
 	stringstyle=\color{codepurple},
 	basicstyle=\ttfamily\scriptsize,
 	breakatwhitespace=false,         
 	breaklines=true,                 
 	captionpos=b,                    
 	keepspaces=true,                 
 	numbers=left,                    
 	numbersep=5pt,                  
 	showspaces=false,                
 	showstringspaces=false,
 	showtabs=false,                  
 	tabsize=4
 }
 
 \lstset{style=mystyle}
 
 
\begin{document}
 \begin{titlepage}
	
	\begin{frame}[t]
		\raisebox{-10mm}[10pt][0pt]{%
			\makebox[\textwidth][c]{\includegraphics[width=\textwidth]{institute.jpg}}
		}
	\end{frame}\\
	\large
	\centering{
		\vspace{5mm}\\
		Міністерство освіти і науки України \\
		Національний технічний університет України \\
		"Київський політехнічний інститут імені Ігоря Сікорського"\\
		Фізико-технічний інститут
	}
	\vspace{3cm}\\
	\centering{
		{\Huge\textbf{Операційні системи} \\ \vspace{0.15cm}}
		{\huge Лабораторна №4}
	}
	
	\vspace{8cm}
	\Large
	\begin{flushright}
		Виконав:\\
		Студент групи ФБ-82\\
		\textbf{Козачок Вячеслав}\\
		Перевірив: \\
		Кіреєнко О.В.
	\end{flushright}
	\vfill
	
	\centering{Київ - 2020}
	
\end{titlepage}

\section*{Завдання до виконання}
\large
\begin{enumerate}
\item Перегляньте список процесів користувача (вас).
\item Перегляньте повний список процесів, запущених у системі. При цьому
гарантуйте збереження інформації від "утікання" з екрана (якщо
процесів багато). Зверніть увагу на ієрархію процесів. Простежте через
поля PID і PPID всю ієрархію процесів тільки-но запущеної вами
команди, починаючи з початкового процесу init. Зверніть увагу на
формування інших полів виводу.
\item Запустіть ще одну оболонку shell. Перегляньте повний список
процесів, запущених вами, при цьому зверніть увагу на ієрархію
процесів і на їхній зв'язок з терміналом. Використовуючи команду
kill, завершіть роботу в цій оболонці.
\item Перегляньте список задач у системі і проаналізуйте їхній стан.
\item Запустіть фоновий процес командою\\
\texttt{find / -name ``*c*'' -print > file 2> /dev/null \&}
\item Визначте його номер. Відправте сигнал призупинення процесу.
Перегляньте список задач у системі і проаналізуйте їхній стан.
Продовжить виконання процесу. Знову перегляньте список задач у
системі і проаналізуйте його зміну. Переведіть процес в активний
режим, а потім знову у фоновий. Запустіть цей процес із пріоритетом 5.
\item Виведіть на екран список усіх процесів, запущених не користувачем
root.
\item Організуйте виведення на екран календаря <2010+Noваріанту> року
через 1 хвилину після поточного моменту часу.
\item Організуйте періодичне (щоденне) видалення в домашньому каталозі
усіх файлів з розширенням *.profilebak і *.profiletmp.

\end{enumerate}

\newpage

\section*{Task 1}\vspace{-5mm}
		\begin{lstlisting}[language=BASH]
[eski@eski-pc OS]$ ps -u $(whoami)
PID TTY          TIME CMD
1506 ?        00:00:00 systemd
1507 ?        00:00:00 (sd-pam)
1518 ?        00:00:01 kwalletd5
1519 ?        00:00:00 startplasma-x11
1529 ?        00:00:05 dbus-daemon
1543 ?        00:00:00 start_kdeinit
1544 ?        00:00:00 kdeinit5
1560 ?        00:00:01 klauncher
		\end{lstlisting}

\section*{Task 2}\vspace{0mm}

Show all processes and their PPID and PID
\begin{lstlisting}[language=BASH]
[eski@eski-pc OS]$ ps axjf
PPID     PID    PGID     SID TTY        TPGID STAT   UID   TIME COMMAND
0       2       0       0 ?             -1 S        0   0:00 [kthreadd]
2       3       0       0 ?             -1 I<       0   0:00  \_ [rcu_gp]
2       4       0       0 ?             -1 I<       0   0:00  \_ [rcu_par_gp]
2       8       0       0 ?             -1 I<       0   0:00  \_ [mm_percpu_wq]
2       9       0       0 ?             -1 S        0   0:02  \_ [ksoftirqd/0]
2      10       0       0 ?             -1 S        0   0:00  \_ [rcuc/0]
2      11       0       0 ?             -1 I        0   0:06  \_ [rcu_preempt]
2      12       0       0 ?             -1 S        0   0:00  \_ [rcub/0]
2      13       0       0 ?             -1 S        0   0:00  \_ [migration/0]
...
\end{lstlisting}

Show process "init"
	\begin{lstlisting}[language=BASH]
[eski@eski-pc OS]$ ps axjf | grep -A 7 "init" | head -n 9
0       1       1       1 ?             -1 Ss       0   0:01 /sbin/init
1     348     348     348 ?             -1 Ss       0   0:00 /usr/bin/lvmetad -f
1     380     380     380 ?             -1 Ss       0   0:09 /usr/sbin/haveged -w 1024 -v 1 --Foreground
1     381     381     381 ?             -1 Ss       0   0:02 /usr/lib/systemd/systemd-journald
1     382     382     382 ?             -1 Ss       0   0:02 /usr/lib/systemd/systemd-udevd
1     985     985     985 ?             -1 Ssl    979   0:00 /usr/lib/systemd/systemd-timesyncd
1     989     989     989 ?             -1 Ss     975   0:00 avahi-daemon: running [eski-pc.local]
989    1042     989     989 ?             -1 S      975   0:00  \_ avahi-daemon: chroot helper
--
\end{lstlisting}

Show processes where parent process is init (PPID == 1)
\begin{lstlisting}[language=BASH]
[eski@eski-pc OS]$ ps axjf | head -n 1; ps axjf | grep "^      1"      
PPID     PID    PGID     SID TTY        TPGID STAT   UID   TIME COMMAND
1     348     348     348 ?             -1 Ss       0   0:00 /usr/bin/lvmetad -f
1     380     380     380 ?             -1 Ss       0   0:09 /usr/sbin/haveged -w 1024 -v 1 --Foreground
1     381     381     381 ?             -1 Ss       0   0:02 /usr/lib/systemd/systemd-journald
1     382     382     382 ?             -1 Ss       0   0:02 /usr/lib/systemd/systemd-udevd
1     985     985     985 ?             -1 Ssl    979   0:00 /usr/lib/systemd/systemd-timesyncd
1     989     989     989 ?             -1 Ss     975   0:00 avahi-daemon: running [eski-pc.local]
1     990     990     990 ?             -1 Ss       0   0:00 /usr/lib/bluetooth/bluetoothd
1     991     991     991 ?             -1 Ss       0   0:00 /usr/bin/crond -n
...
	\end{lstlisting}

\section*{Task 3}\vspace{0mm}
Show processes in new terminal
\begin{lstlisting}[language=BASH]
[eski@eski-pc ~]$ ps a -u eski
PID TTY      STAT   TIME COMMAND
1159 tty1     Ssl+   6:30 /usr/lib/Xorg -nolisten tcp -auth /var/run/sddm/{54bf40ef-e9d2-4ae4-9041-2434e7c0fbe7} -bac
1506 ?        Ss     0:00 /usr/lib/systemd/systemd --user
1507 ?        S      0:00 (sd-pam)
...
   9695 pts/5    Ss+    0:00 /bin/bash
10236 ?        Sl     0:05 /usr/bin/konsole
10254 pts/6    Ss     0:00 /bin/bash #this terminal
10360 pts/6    R+     0:00 ps a -u eski
\end{lstlisting}

Exit this terminal
\begin{lstlisting}[language=BASH]
[eski@eski-pc ~]$ kill -9 10254

Warning: Program '/bin/bash' crashed.
\end{lstlisting}

\section*{Task 4}\vspace{0mm}

See some tasks and jobs
\begin{lstlisting}[language=BASH]
[eski@eski-pc OS]$ jobs
#empty
\end{lstlisting}
\begin{lstlisting}[language=BASH]
[eski@eski-pc OS]$ ps
PID TTY          TIME CMD
9695 pts/5    00:00:00 bash
10937 pts/5    00:00:00 ps

\end{lstlisting}

\section*{Task 5}\vspace{0mm}
Launch process(job) \texttt{find} in background. 
\begin{lstlisting}[language=BASH]
eski@eski-pc OS]\$ find / -name "*c*" -print > file2 2> /dev/null \&
[1] 11034

\end{lstlisting}

\section*{Task 6}\vspace{-3mm}
\begin{lstlisting}[language=BASH]
[eski@eski-pc OS]\$ kill -19 11034

[eski@eski-pc OS]\$ ps
PID TTY          TIME CMD
9695 pts/5    00:00:00 bash
11034 pts/5    00:00:03 find
11040 pts/5    00:00:00 ps

[1]+  Stopped                 find / -name "*c*" -print > file2 2> /dev/null

[eski@eski-pc OS]$ kill -18 11034

[eski@eski-pc OS]$ ps
PID TTY          TIME CMD
9695 pts/5    00:00:00 bash
11034 pts/5    00:00:04 find
11047 pts/5    00:00:00 ps

[eski@eski-pc OS]$ fg
find / -name "*c*" -print > file 2> /dev/null
^Z
[1]+  Stopped                 find / -name "*c*" -print > file 2> /dev/null

[eski@eski-pc OS]$ bg
[1]+ find / -name "*c*" -print > file 2> /dev/null &

[eski@eski-pc OS]\$ renice -n 5 -p 11034
11908 (process ID) old priority 0, new priority 5

\end{lstlisting}

\section*{Task 7}\vspace{-3mm}
\begin{lstlisting}[language=BASH]
[eski@eski-pc OS]$ ps aux | grep -v "^root"

\end{lstlisting}
\newpage
\section*{Task 8}\vspace{-3mm}
\begin{lstlisting}[language=BASH]
[eski@eski-pc OS]$ echo "cal $((2010+9)) > ~/calendar2019" | at -m now +1 minute
warning: commands will be executed using /bin/sh
job 2 at Thu Mar 19 16:28:00 2020

[eski@eski-pc OS]$ cat ~/calendar2019 
2019                               

	January               February                 March       
Su Mo Tu We Th Fr Sa   Su Mo Tu We Th Fr Sa   Su Mo Tu We Th Fr Sa
1  2  3  4  5                   1  2                   1  2
6  7  8  9 10 11 12    3  4  5  6  7  8  9    3  4  5  6  7  8  9
13 14 15 16 17 18 19   10 11 12 13 14 15 16   10 11 12 13 14 15 16
20 21 22 23 24 25 26   17 18 19 20 21 22 23   17 18 19 20 21 22 23
27 28 29 30 31         24 25 26 27 28         24 25 26 27 28 29 30
31                  
	April                   May                   June        
Su Mo Tu We Th Fr Sa   Su Mo Tu We Th Fr Sa   Su Mo Tu We Th Fr Sa
1  2  3  4  5  6             1  2  3  4                      1
7  8  9 10 11 12 13    5  6  7  8  9 10 11    2  3  4  5  6  7  8
14 15 16 17 18 19 20   12 13 14 15 16 17 18    9 10 11 12 13 14 15
21 22 23 24 25 26 27   19 20 21 22 23 24 25   16 17 18 19 20 21 22
28 29 30               26 27 28 29 30 31      23 24 25 26 27 28 29
30                  
	July                  August                September     
Su Mo Tu We Th Fr Sa   Su Mo Tu We Th Fr Sa   Su Mo Tu We Th Fr Sa
1  2  3  4  5  6                1  2  3    1  2  3  4  5  6  7
7  8  9 10 11 12 13    4  5  6  7  8  9 10    8  9 10 11 12 13 14
14 15 16 17 18 19 20   11 12 13 14 15 16 17   15 16 17 18 19 20 21
21 22 23 24 25 26 27   18 19 20 21 22 23 24   22 23 24 25 26 27 28
28 29 30 31            25 26 27 28 29 30 31   29 30               

	October               November               December      
Su Mo Tu We Th Fr Sa   Su Mo Tu We Th Fr Sa   Su Mo Tu We Th Fr Sa
1  2  3  4  5                   1  2    1  2  3  4  5  6  7
6  7  8  9 10 11 12    3  4  5  6  7  8  9    8  9 10 11 12 13 14
13 14 15 16 17 18 19   10 11 12 13 14 15 16   15 16 17 18 19 20 21
20 21 22 23 24 25 26   17 18 19 20 21 22 23   22 23 24 25 26 27 28
27 28 29 30 31         24 25 26 27 28 29 30   29 30 31
\end{lstlisting}

	
\section*{Task 9}\vspace{-3mm}
\begin{lstlisting}[language=BASH]
[eski@eski-pc OS]$ crontab -e
#add some lines
[eski@eski-pc OS]$ crontab -l
@daily rm ~/*.bak;
@daily rm ~/*.tmp;
\end{lstlisting}

\vspace{5mm}
\section*{Висновки}
В цій лабораторній роботі я навчився працювати з процесами, а саме змінювати пріорітетність процесу, запускати процес у фоновому режимі, переводити процес у активний режим, закінчувати життя процесу.
Також навчився виконувати завдання з заданим інтервалом за допомогою демона \texttt{crond} та команди \texttt{сrontab -e} та виконувати команди одноразово за допомогою команди \texttt{at}

\end{document}
