 \documentclass{article}

\usepackage{fancyhdr}
\usepackage{cmap}
\usepackage[T2A]{fontenc}
\usepackage[utf8]{inputenc}
\usepackage[english,russian,ukrainian]{babel}
\usepackage{pythontex}
\usepackage{indentfirst}
\usepackage{graphicx}
\usepackage{hyperref}
\hypersetup{
	colorlinks,
	citecolor=black,
	linkcolor=black,
	filecolor=black,
	urlcolor=black
}

\usepackage{geometry}
\geometry{
	a4paper,
	total={165mm,247mm},
	left=20mm,
	top=30mm,
}
\usepackage{titlesec}
\usepackage{fancyhdr}

\pagestyle{fancy}
\fancyhf{}
\renewcommand{\headrulewidth}{0.5pt}
%\renewcommand{\footrulewidth}{0.1pt}
\rhead{\LaTeX}
\lhead{Вячеслав Козачок}

\cfoot{\thepage}
%\titleformat{\section}
%{\normalfont\Large\bfseries}{\thesection}{1em}{}

%\titleformat*{\subsection}{\large\bfseries}

 \usepackage{listings}
 \usepackage{xcolor}
 
 \definecolor{codegreen}{rgb}{0,0.6,0}
 \definecolor{codegray}{rgb}{0.5,0.5,0.5}
 \definecolor{codepurple}{rgb}{0.58,0,0.82}
 \definecolor{backcolour}{rgb}{0.99,0.99,0.99}
 
 
 \lstdefinestyle{mystyle}{
 	backgroundcolor=\color{backcolour},   
 	commentstyle=\color{codegreen},
 	keywordstyle=\color{magenta},
 	numberstyle=\tiny\color{codegray},
 	stringstyle=\color{codepurple},
 	basicstyle=\ttfamily\footnotesize,
 	breakatwhitespace=false,         
 	breaklines=true,                 
 	captionpos=b,                    
 	keepspaces=true,                 
 	numbers=left,                    
 	numbersep=5pt,                  
 	showspaces=false,                
 	showstringspaces=false,
 	showtabs=false,                  
 	tabsize=4
 }
 
 \lstset{style=mystyle}
 
 
 \begin{document}

 	\begin{titlepage}
	
	\begin{frame}[t]
		\raisebox{-10mm}[10pt][0pt]{%
			\makebox[\textwidth][c]{\includegraphics[width=\textwidth]{institute.jpg}}
		}
	\end{frame}\\
	\large
	\centering{
		\vspace{5mm}\\
		Міністерство освіти і науки України \\
		Національний технічний університет України \\
		"Київський політехнічний інститут імені Ігоря Сікорського"\\
		Фізико-технічний інститут
	}
	\vspace{3cm}\\
	\centering{
		{\Huge\textbf{Операційні системи} \\ \vspace{0.15cm}}
		{\huge Лабораторна №4}
	}
	
	\vspace{8cm}
	\Large
	\begin{flushright}
		Виконав:\\
		Студент групи ФБ-82\\
		\textbf{Козачок Вячеслав}\\
		Перевірив: \\
		Кіреєнко О.В.
	\end{flushright}
	\vfill
	
	\centering{Київ - 2020}
	
\end{titlepage}
 	\newpage
 	
 	\section*{Варіант 3}
 	\begin{enumerate} 
\item Написати сценарій для оболонки bash згідно таких вимог:
 	Сценарій приймає 3 параметри командного рядка. Перший параметр —
 	ім’я каталогу, в якому (і в підкаталогах якого рекурсивно) треба
 	здійснювати пошук. Другий параметр — шаблон пошуку. За
 	відсутності двох параметрів сценарій повинен виводити рядок, що
 	описує коректний формат виклику сценарію. Третій параметр —
 	необов’язковий — це ім’я файлу, що буде створений.
 	\item Сценарій шукає у заданому каталозі і його підкаталогах усі файли, які
 	містять послідовність байт, що відповідає шаблону пошуку, і формує їх
 	список. Список повинен містити ім’я файлу, повний шлях до каталогу,
 	тип файлу (див. команду file), розмір файлу у байтах, і рядок, що
 	відповідає шаблону.
 	\item Якщо третій параметр виклику сценарію (ім’я файлу) був заданий, то
 	список має бути виведений у цей файл. Якщо параметр заданий не
 	був — на цьому етапі сценарій повинен запросити ввести його. В обох
 	випадках якщо такий файл для виводу вже існує, сценарій повинен
 	видати запит, що робити далі — переписати вміст файлу, дописати нові
 	дані у кінець файлу, або скасувати операцію.
 	38Операційні системи – комп’ютерний практикум
 	\item Сценарій повинен коректно відпрацьовувати помилки, такі як
 	некоректні імена файлів і шаблон пошуку, відсутність заданого
 	каталогу, помилки доступу (зокрема, відсутність права доступу до
 	певних файлів), відсутність жодних файлів, що відповідають заданому
 	шаблону. При цьому сценарій повинен видавати діагностику помилок.
 \end{enumerate}
 	\section*{Робота}
\begin{lstlisting}[language=BASH]
#!/bin/bash
usage="script -d directory -t template -f filename"

checkargs() {
	if [[ $OPTARG =~ ^-[d,t,f]$ ]]
	then
		echo "Unknown argument $OPTARG for option $option"
		exit 1
	else
		echo "lol"
	fi
}

directory=
template=
filename=
while getopts "d:t:f:" option 
do
	case "$option" in 
		d) 	checkargs
		directory=${OPTARG};; 
		t) 	checkargs 
		template=${OPTARG};;
		f) 	checkargs 
		filename=${OPTARG};;
	esac
done


if [ -d "$directory" ];
then 
	echo "$directory valid"
else
	echo "Such directory does not exists."
	exit 1
fi 

if [[ $directory == "" || $template == "" ]];
then
	echo " [!] Usage: $usage"
	exit 1
fi

if [[ $filename == "" ]];
then
	read -p "Please enter the name of the file: "
	filename=$REPLY
fi

if [[ -f $filename ]];
then
	printf "This file: $filename already exists.\n What do you want to do?\n  [1] Rewrite file\n  [2] Append to file\n  [3] exit\n  Choice:"
	read  
	if [[ $REPLY == "1" ]];	then
		rm $filename
		touch $filename
	fi
	
	if [[ $REPLY == "3" ]]; then
		exit 1
	fi
	touch /tmp/$filename.err
	touch /tmp/$filename.bin
fi

printf '\n'

grep -nrwi $directory -e "$template" 2> /tmp/$filename.err > "/tmp/$filename.tmp"

while IFS="" read -r p || [ -n "$p" ]
do 
	if [[ "$p" =~ .*"matches"*. ]];
	then
		echo "$p" > "/tmp/$filename.bin"
	fi
done < /tmp/$filename.tmp 

sed -i -e '/matches/d' /tmp/$filename.tmp

while IFS="" read -r p || [ -n "$p" ]
do
	name=$(cut -d: -f1 <<< "$p")
	text=$(cut -d: -f3 <<< "$p")
	echo "$(basename "$name"):$(realpath $name):$(file $name | cut -d: -f2):bytes $(wc -c $name | cut -d" " -f1):text $text" >> $filename

done < "/tmp/$filename.tmp"

while IFS="" read -r p || [ -n "$p" ]
do	
	name=$(cut -d" " -f3 <<< "$p")
	echo "$(basename "$name"):$(realpath $name):$(file $name | cut -d: -f2):bytes $(wc -c $name | cut -d" " -f1):binary file" >> $filename
done < /tmp/$filename.bin

rm "/tmp/$filename.tmp"
rm "/tmp/$filename.err"
rm "/tmp/$filename.bin"

\end{lstlisting}

\section*{Висновки}
\large
Навчився писати скрипти для bash. Оволодів практичними навичками професійної роботи з
командною оболонкою shell – використанням змінних і створенням командних файлів.

\end{document}