\documentclass{article}

\usepackage{cmap}
\usepackage[T2A]{fontenc}
\usepackage[utf8]{inputenc}
\usepackage[english,ukrainian,russian]{babel}
\usepackage{hyperref}
\usepackage{indentfirst}
\usepackage{verbatim}
\usepackage{fancyhdr}
\usepackage{multirow}
\usepackage{tabularx}
\usepackage[absolute]{textpos}

\pagestyle{fancy}
\fancyhf{}
\rfoot{Page \thepage}
\rhead{\LaTeX}
\lhead{Вячеслав Козачок ФБ-82}
\renewcommand{\headrulewidth}{1pt}
\renewcommand{\footrulewidth}{1pt}

\begin{document}


	\centering{
		Міністерство освіти і науки України \\
		Національний технічний університет України
		"Київський політехнічний інститут імені Ігоря Сікорського"\\
		Фізико-технічний інститут
	}
	\vspace{4cm}\\
	\centering{
	{\LARGE	\textbf{Операційні системи} \\ \vspace{0.15cm}}
		{\Large Лабораторна №1}
}

\vspace{7cm}
\begin{flushright}
	Виконав: студент групи ФБ-82\\
	Козачок Вячеслав\\
	Перевірив: \\
	Кирієнко О.В.
\end{flushright}
\vfill
	
\centering{Київ - 2020}
\newpage
	\noindent \section*{Номер варіанту: 9}
	\indent \textbf{Завдання до виконання}
	\begin{enumerate}
		\item Завантажтеся в систему під вашим користувацьким ім'ям.
		\item Поміняйте ваш пароль. Ваш новий пароль повинен включати в себе як
		частину номер Вашої залікової книжки.
		\item Виведіть системну дату.
		\item Підрахуйте кількість рядків у файлі: 
\verb|/etc/protocols|
		\item Виведіть на екран вміст відповідного файлу.
		\item Виведіть календар на <1995+Noваріанту> рік.
		\item Виведіть календар на 1752 рік. Чи не помічаєте що-небудь цікаве у
		вересні? Поясніть.
		\item Визначте, хто ще завантажений у систему.
		\item Наберіть команду ping. Поясніть результат.
		\item Скопіюйте (скопіюйте, а не перемістіть, бо система перестане
		працювати коректно!) файли: \verb|/bin/login та /bin/sh|
		\item Створіть каталог \verb|lab_1| 
		\item Скопіюйте в нього з вашого домашнього каталогу копію файлу 1, яку
		ви отримали в п.10. Перемістіть в
		цей каталог з вашого домашнього каталогу копію файлу 2, яку ви
		отримали в п.10
		\item Перейдіть у свій домашній каталог і переконайтеся в тому, що все
		зроблено правильно.
		\item Створіть каталог \verb|lab_1_<необхідна_команда>_<No_варіанту>| і перейдіть в нього.
		\item Скопіюйте в каталог \verb|lab_1_<необхідна_команда>_<No_варіанту>| файл з п.4 під ім'ям
		\verb|n<необхідна_команда>ім’я вихідного файлу> |.
		\item За допомогою команд cat і less перегляньте його вміст.
		\item Перейдіть у свій домашній каталог.
		\item Видаліть каталог \verb|lab_1_<необхідна_команда>Noваріанту>|.
	\end{enumerate}
\clearpage
\section*{Виконання роботи}
\begin{enumerate}
	\item Login topshop\\Password: **********
	\item \begin{verbatim}
	[topshop@eski-pc ~]$ passwd
	Changing password for topshop.
	Current password: 
	New password: 
	Retype new password: 
	passwd: password updated successfully
	\end{verbatim}
	\item \begin{verbatim}
	[topshop@eski-pc ~]$ date
	Fri 14 Feb 2020 01:41:52 AM EET
	\end{verbatim}
	\item \begin{verbatim}
	[topshop@eski-pc \~]\$ wc -l /etc/protocols 
	137 /etc/protocols
	\end{verbatim}
	
	\item \begin{verbatim}
	[topshop@eski-pc ~]$ cat /etc/protocols 
	# Full data: /usr/share/iana-etc/protocol-numbers.iana
	hopopt         0 HOPOPT
	icmp           1 ICMP
	igmp           2 IGMP
	ggp            3 GGP
	ipv4           4 IPv4
	............................
	wesp         141 WESP
	rohc         142 ROHC
	ethernet     143 Ethernet
	reserved     255 Reserved
	
	\end{verbatim}
	\item \begin{verbatim}
	[topshop@eski-pc ~]$ cal 2004
	2004                               
	
	January               February                 March       
	Su Mo Tu We Th Fr Sa   Su Mo Tu We Th Fr Sa   Su Mo Tu We Th Fr Sa
	1  2  3    1  2  3  4  5  6  7       1  2  3  4  5  6
	4  5  6  7  8  9 10    8  9 10 11 12 13 14    7  8  9 10 11 12 13
	11 12 13 14 15 16 17   15 16 17 18 19 20 21   14 15 16 17 18 19 20
	18 19 20 21 22 23 24   22 23 24 25 26 27 28   21 22 23 24 25 26 27
	25 26 27 28 29 30 31   29                     28 29 30 31         
	
	......................
	
	\end{verbatim}
	\clearpage
	\item \begin{verbatim}
	[topshop@eski-pc ~]$ cal 1752
	1752                               
	
	................               
	July                  August                September     
	Su Mo Tu We Th Fr Sa   Su Mo Tu We Th Fr Sa   Su Mo Tu We Th Fr Sa
	1  2  3  4                      1          1  2 14 15 16
	5  6  7  8  9 10 11    2  3  4  5  6  7  8   17 18 19 20 21 22 23
	12 13 14 15 16 17 18    9 10 11 12 13 14 15   24 25 26 27 28 29 30
	19 20 21 22 23 24 25   16 17 18 19 20 21 22                       
	26 27 28 29 30 31      23 24 25 26 27 28 29                       
	30 31                                      
	.................
	
	\end{verbatim}
	\item \begin{verbatim}
	[topshop@eski-pc eski]$ users
	topshop eski
	
	\end{verbatim}
	\item \begin{verbatim}
	[topshop@eski-pc eski]$ ping
	ping: usage error: Destination address required
	
	[topshop@eski-pc eski]$ ping 8.8.8.8
	PING 8.8.8.8 (8.8.8.8) 56(84) bytes of data.
	64 bytes from 8.8.8.8: icmp_seq=1 ttl=56 time=16.1 ms
	64 bytes from 8.8.8.8: icmp_seq=2 ttl=56 time=16.6 ms
	64 bytes from 8.8.8.8: icmp_seq=3 ttl=56 time=14.7 ms
	^C
	--- 8.8.8.8 ping statistics ---
	3 packets transmitted, 3 received, 0% packet loss, time 2003ms
	rtt min/avg/max/mdev = 14.724/15.808/16.624/0.798 ms
	
	\end{verbatim}
	\item \begin{verbatim}
	[topshop@eski-pc ~]$ cp /bin/login .
	[topshop@eski-pc ~]$ cp /bin/sh .
	[topshop@eski-pc ~]$ ls -l
	total 928
	-rwxr-xr-x 1 topshop topshop  42840 Feb 14 01:53 login
	-rwxr-xr-x 1 topshop topshop 903440 Feb 14 01:54 sh
	
	\end{verbatim}
	\item \begin{verbatim}
	[topshop@eski-pc ~]$ mkdir lab_1
	
	\end{verbatim}
	\item \begin{verbatim}
	[topshop@eski-pc ~]$ cp login lab_1/my_login
	[topshop@eski-pc ~]$ mv sh lab_1/my_sh
	
	\end{verbatim}
	\item \begin{verbatim}
	[topshop@eski-pc ~]$ ls
	lab_1  login
	[topshop@eski-pc ~]$ ls lab_1/
	my_login  my_sh
	
	\end{verbatim}
	\item \begin{verbatim}
	[topshop@eski-pc ~]$ mkdir lab_1_9
	[topshop@eski-pc ~]$ cd lab_1_9/
	
	\end{verbatim}
	\item \begin{verbatim}
	[topshop@eski-pc lab_1_9]$ cp /etc/protocols nprotocols
	
	\end{verbatim}
	\item \begin{verbatim}
	[topshop@eski-pc lab_1_9]$ cat nprotocols 
	# Full data: /usr/share/iana-etc/protocol-numbers.iana
	hopopt         0 HOPOPT
	icmp           1 ICMP
	igmp           2 IGMP
	..........
	 
	[topshop@eski-pc lab_1_9]$ less nprotocols
	------FILE-----
	\end{verbatim}
	\item \begin{verbatim}
	[topshop@eski-pc lab_1_9]$ cd ~
	
	\end{verbatim}
	\item \begin{verbatim}
	[topshop@eski-pc ~]$ rm -rf lab_1_9
	[topshop@eski-pc ~]$ ls 
	lab_1  login
	
	\end{verbatim}
\end{enumerate}


\end{document}
