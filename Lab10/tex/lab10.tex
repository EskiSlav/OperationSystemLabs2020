\documentclass{article}

\usepackage{fancyhdr}
\usepackage{cmap}
\usepackage[T2A]{fontenc}
\usepackage[utf8]{inputenc}
\usepackage[english,russian,ukrainian]{babel}
\usepackage{indentfirst}
\usepackage{graphicx}
\usepackage{hyperref}
\usepackage{titlesec}
\usepackage{fancyhdr}
\usepackage{geometry}

\hypersetup{
	colorlinks,
	citecolor=black,
	linkcolor=black,
	filecolor=black,
	urlcolor=black
}

\geometry{
	a4paper,
	total={165mm,247mm},
	left=20mm,
	top=30mm,
}

\pagestyle{fancy}
\fancyhf{}
\renewcommand{\headrulewidth}{0.5pt}
%\renewcommand{\footrulewidth}{0.1pt}
\rhead{\LaTeX}
\lhead{Вячеслав Козачок}

\cfoot{\thepage}
%\titleformat{\section}
%{\normalfont\Large\bfseries}{\thesection}{1em}{}

%\titleformat*{\subsection}{\large\bfseries}

 \usepackage{listings}
 \usepackage{xcolor}
 
 \definecolor{codegreen}{rgb}{0,0.6,0}
 \definecolor{codegray}{rgb}{0.5,0.5,0.5}
 \definecolor{codepurple}{rgb}{0.58,0,0.82}
 \definecolor{backcolour}{rgb}{0.99,0.99,0.99}
 \definecolor{codeblue}{rgb}{0.1,0.1,0.99}
 
 \lstdefinestyle{mystyle}{
 	backgroundcolor=\color{backcolour},   
 	commentstyle=\color{codegreen},
 	keywordstyle=\color{codeblue},
 	numberstyle=\tiny\color{codegray},
 	stringstyle=\color{codepurple},
 	basicstyle=\ttfamily\footnotesize,
 	breakatwhitespace=false,         
 	breaklines=true,                 
 	captionpos=b,                    
 	keepspaces=true,                 
 	numbers=left,                    
 	numbersep=5pt,                  
 	showspaces=false,                
 	showstringspaces=false,
 	showtabs=false,                  
 	tabsize=4
 }
 
 \lstset{style=mystyle}
 \newcommand{\nitem}{\item \vspace{-2mm}}
 
\begin{document}
 \begin{titlepage}
	
	\begin{frame}[t]
		\raisebox{-10mm}[10pt][0pt]{%
			\makebox[\textwidth][c]{\includegraphics[width=\textwidth]{institute.jpg}}
		}
	\end{frame}\\
	\large
	\centering{
		\vspace{5mm}\\
		Міністерство освіти і науки України \\
		Національний технічний університет України \\
		"Київський політехнічний інститут імені Ігоря Сікорського"\\
		Фізико-технічний інститут
	}
	\vspace{3cm}\\
	\centering{
		{\Huge\textbf{Операційні системи} \\ \vspace{0.15cm}}
		{\huge Лабораторна №4}
	}
	
	\vspace{8cm}
	\Large
	\begin{flushright}
		Виконав:\\
		Студент групи ФБ-82\\
		\textbf{Козачок Вячеслав}\\
		Перевірив: \\
		Кіреєнко О.В.
	\end{flushright}
	\vfill
	
	\centering{Київ - 2020}
	
\end{titlepage}
\large
\section*{Варіант 9}

Нехай процес, який відкрив N файлів, перед породженням процесу-нащадка за допомогою системного виклику \texttt{fork()} закриває K<N файлів.
Процес-нащадок відразу після породження закриває M<N-K файлів і через
деякий час завершується (в цей час процес-предок очікує його завершення).
Розробити програму, яка демонструвала б динаміку зміни даних в системі
керування введенням-виведенням ОС Linux (таблиці відкритих файлів і
масиви файлових дескрипторів процесів). Наприклад, сценарій програми
може бути таким:

\begin{enumerate}
\nitem відкриття процесом-предком стандартних файлів введення-виведення і
чотирьох призначених для користувача файлів для зчитування;
\nitem закриття процесом-предком двох призначених для користувача файлів;
\nitem процес-предок породжує процес, який успадковує таблиці файлів і
відкритих файлів процесу-предка;
\nitem завершується процес-нащадок.
Після кожного з етапів друкуються таблиці відкритих файлів і масиви
файлових дескрипторів для обох процесів.
\end{enumerate}

\section*{Code}
\begin{lstlisting}[language=C++]
#include <sys/types.h>
#include <sys/stat.h>
#include <sys/wait.h>
#include <string.h>
#include <unistd.h>
#include <stdio.h>
#include <fcntl.h>

#define FILES 8

int * fd = new int[FILES];


void print_descriptors(int * fd, int print_offset=1)
{
    for(int i = 0; i < print_offset; i++) printf(" ");

    int opened_files = 0;
    for(int i = 0; i < FILES; i++)
    {
        struct stat buff;
        if (fstat(fd[i], &buff) == -1)
            continue;
        else
        {
            opened_files++;
            printf("%d ", fd[i]);
        }
    }
    printf("\n");
    for(int i = 0; i < print_offset; i++) printf(" ");
    printf("Total opened files:%d\n\n", opened_files);
}

int main() {
    // Array of file descriptors
    fd[0] = open("/dev/stdin", O_RDWR); 
    fd[1] = open("/dev/stdout", O_RDWR); 
    fd[2] = open("/dev/stderr", O_RDWR); 
    fd[3] = open("./files/dum.txt", O_RDWR); 
    fd[4] = open("./files/rob.txt", O_RDWR); 
    fd[5] = open("./files/ann.txt", O_RDWR); 
    fd[6] = open("./files/kate.txt", O_RDWR); 
    fd[7] = open("./files/file.txt", O_RDWR); 
    

    printf(" - PID:%d All opened descriptors:\n", getpid());
    print_descriptors(fd);

    printf(" - PID:%d Closing some descriptors ... \n", getpid());
    close(fd[2]);
    close(fd[3]);


    printf(" - PID:%d Show descriptors again:\n", getpid());
    print_descriptors(fd);

    printf(" - PID:%d Creating Child Process...\n", getpid());
    pid_t pID = fork();

    switch(pID)
    {
        case 0:
            // Child process section
            printf(" - - PID:%d Child process showing descriptors...\n", getpid());
            print_descriptors(fd, 5);  

            printf(" - - PID:%d Child process closing some descriptors...\n", getpid());
            close(fd[6]);
            close(fd[5]);

            printf(" - - PID:%d Show descriptors in child process after closing:\n",
             getpid());
            print_descriptors(fd, 5);

            printf(" - - PID:%d Child process EXITING...\n", getpid());
            break;

        case -1:
            // Error Section
            printf("Something bad happened when creating process\n");
            perror("fork");
            break;

        default:
            // Parent process section
            wait(&pID);
            printf(" - PID:%d Parent process waited for child and now prints descriptors again\n", getpid());
            print_descriptors(fd);
            printf(" - PID:%d Parent process EXITING...\n", getpid());
            break;
    }
    

}
\end{lstlisting}
\newpage
\section*{Output}
\begin{lstlisting}[language=BASH]
 - PID:43563 All opened descriptors:
 3 4 5 6 7 8 9 10 
 Total opened files:8

 - PID:43563 Closing some descriptors ... 
 - PID:43563 Show descriptors again:
 3 4 7 8 9 10 
 Total opened files:6

 - PID:43563 Creating Child Process...
 - - PID:43564 Child process showing descriptors...
     3 4 7 8 9 10 
     Total opened files:6

 - - PID:43564 Child process closing some descriptors...
 - - PID:43564 Show descriptors in child process after closing:
     3 4 7 10 
     Total opened files:4

 - - PID:43564 Child process EXITING...
 - PID:43563 Parent process waited for child and now prints descriptors again
 3 4 7 8 9 10 
 Total opened files:6

 - PID:43563 Parent process EXITING...

\end{lstlisting}
\newpage
\section*{Висновки}
В цій лабораторній роботі навчився працювати з файловими дескрипторами. Відкривати їх, закривати дочірнім процессом, але як ми можемо бачити, коли ми закриваємо дочірнім процессом дескриптор, він залишається відкритим у батьківському. Так виходить тому що до дочірнього процесу потрапляють лише копії дискрипторів, і тому у батьківському вони залишаються.


У функції \verb|print_descriptors| ми виводимо тільки відкриті дескриптори, адже якщо \verb|fstat()| повертає -1 то файл не відкритий і дескриптор є не дійсним і тому ми не виводимо дескриптор, а якщо 0, то ми можемо працювати з файлом, отже виводимо дескриптор. Це і реалізовано у функції.


\end{document}