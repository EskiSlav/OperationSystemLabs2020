\documentclass{article}

\usepackage{fancyhdr}
\usepackage{cmap}
\usepackage[T2A]{fontenc}
\usepackage[utf8]{inputenc}
\usepackage[english,russian,ukrainian]{babel}
\usepackage{indentfirst}
\usepackage{graphicx}
\usepackage{hyperref}
\usepackage{titlesec}
\usepackage{fancyhdr}
\usepackage{geometry}

\hypersetup{
	colorlinks,
	citecolor=black,
	linkcolor=blue,
	filecolor=black,
	urlcolor=black
}

\geometry{
	a4paper,
	total={165mm,247mm},
	left=20mm,
	top=30mm,
}

\pagestyle{fancy}
\fancyhf{}
\renewcommand{\headrulewidth}{0.5pt}
%\renewcommand{\footrulewidth}{0.1pt}
\rhead{\LaTeX}
\lhead{Вячеслав Козачок}

\cfoot{\thepage}
%\titleformat{\section}
%{\normalfont\Large\bfseries}{\thesection}{1em}{}

%\titleformat*{\subsection}{\large\bfseries}

 \usepackage{listings}
 \usepackage{xcolor}
 
 \definecolor{codegreen}{rgb}{0,0.6,0}
 \definecolor{codegray}{rgb}{0.5,0.5,0.5}
 \definecolor{codepurple}{rgb}{0.58,0,0.82}
 \definecolor{backcolour}{rgb}{0.99,0.99,0.99}
 \definecolor{codeblue}{rgb}{0.1,0.1,0.99}
 
 \lstdefinestyle{mystyle}{
 	backgroundcolor=\color{backcolour},   
 	commentstyle=\color{codegreen},
 	keywordstyle=\color{codeblue},
 	numberstyle=\tiny\color{codegray},
 	stringstyle=\color{codepurple},
 	basicstyle=\ttfamily\footnotesize,
 	breakatwhitespace=false,         
 	breaklines=true,                 
 	captionpos=b,                    
 	keepspaces=true,                 
 	numbers=left,                    
 	numbersep=5pt,                  
 	showspaces=false,                
 	showstringspaces=false,
 	showtabs=false,                  
 	tabsize=4
 }
 
 \lstset{style=mystyle}
 
 
\begin{document}
 \begin{titlepage}
	
	\begin{frame}[t]
		\raisebox{-10mm}[10pt][0pt]{%
			\makebox[\textwidth][c]{\includegraphics[width=\textwidth]{institute.jpg}}
		}
	\end{frame}\\
	\large
	\centering{
		\vspace{5mm}\\
		Міністерство освіти і науки України \\
		Національний технічний університет України \\
		"Київський політехнічний інститут імені Ігоря Сікорського"\\
		Фізико-технічний інститут
	}
	\vspace{3cm}\\
	\centering{
		{\Huge\textbf{Операційні системи} \\ \vspace{0.15cm}}
		{\huge Лабораторна №4}
	}
	
	\vspace{8cm}
	\Large
	\begin{flushright}
		Виконав:\\
		Студент групи ФБ-82\\
		\textbf{Козачок Вячеслав}\\
		Перевірив: \\
		Кіреєнко О.В.
	\end{flushright}
	\vfill
	
	\centering{Київ - 2020}
	
\end{titlepage}


\section{Розминка.}
\large Стандартна задача виробник-споживач.
C++
\subsection{Input data}
\label{inputdata}
\noindent Lorem ipsum dolor sit amet, consectetur adipiscing elit, sed do eiusmod 
tempor incididunt ut labore et dolore magna aliqua. Nulla facilisi nullam vehicula 
ipsum a arcu cursus vitae. Donec et odio pellentesque diam volutpat commodo. In pellentesque massa

\noindent placerat duis ultricies lacus sed. Enim nec dui nunc mattis enim. Id consectetur purus
ut faucibus pulvinar elementum integer. Pharetra pharetra massa massa ultricies.
Pretium quam vulputate dignissim suspendisse in est ante. Sapien nec sagittis
aliquam malesuada bibendum. Quis hendrerit dolor magna eget est lorem ipsum. Amet cursus sit amet dictum sit.

\noindent Leo integer malesuada nunc vel risus commodo. Arcu non odio euismod lacinia. 
Orci ac auctor augue mauris. Eu non diam phasellus vestibulum lorem. Nibh tortor id aliquet lectus proin nibh.

\noindent Proin fermentum leo vel orci porta non pulvinar neque. Molestie nunc non blandit massa enim nec dui nunc mattis. 

\noindent Habitant morbi tristique senectus et. Mattis rhoncus urna neque viverra justo nec. Ultricies integer quis auctor

\noindent elit sed vulputate mi sit. Justo eget magna fermentum iaculis.

\subsection{Code}

\begin{lstlisting}[language=C++]
#include <stdlib.h>
#include <stdio.h>
#include <pthread.h>
#include <unistd.h>
#include <queue>
#include <semaphore.h>
#include <time.h>
using std::queue;

queue<char*> q;
const int max_size = 5;
FILE * file;
int done = 0;

sem_t lock;
sem_t full;
sem_t empty;

void * produce(void * param)
{
	char * text;
	while(!done)
	{
		sem_wait(&empty);
		sem_wait(&lock);
		text = (char*)malloc(255 * sizeof(char));
		if (fgets(text, 256, file) != NULL)
		{
			q.push(text);
			printf("%s %d [!] Text has been added! \n\n", (char*)param, pthread_self());
		}
		else
		{
			done = 1;
		}
		sem_post(&lock);
		sem_post(&full);
		sleep(rand() % 4); // random sleep;
	}
	return NULL;
}

void * consume(void * param)
{
	while(!done || !q.empty()) // when both conditions is False exit the loop;
	{
		sem_wait(&full);
		sem_wait(&lock);
		if(!q.empty())
		{
			printf("%s: %d [!] Text: \n----------------------------------\n %s-------------------------------------\n",(char*)param , pthread_self(), q.front());
			q.pop();
		}
		sem_post(&lock);
		sem_post(&empty);
		sleep(rand() % 4); // random sleep;
	}
	return NULL;
}

int main(int argv, char * argc[])
{
	srand(time(0));

	sem_init(&lock, 0, 1);
	sem_init(&full, 0, 0);
	sem_init(&empty,0, max_size);
	file = fopen("lorem_ipsum.txt", "r");

	pthread_t cons1, cons2;
	pthread_t prod1, prod2;

	pthread_create(&prod1, NULL, produce, (void*)"producer1");
	pthread_create(&prod2, NULL, produce, (void*)"producer2");
	pthread_create(&cons1, NULL, consume, (void*)"consumer1");
	pthread_create(&cons2, NULL, consume, (void*)"consumer2");

	pthread_join(prod1, NULL);
	pthread_join(prod2, NULL);
	pthread_join(cons1, NULL);
	pthread_join(cons2, NULL);
	
	return 0;
}
\end{lstlisting}
\subsection{Program output}
\normalsize
\begin{verbatim}
producer1 -2053208320 [!] Text has been added! 

consumer2: 2147481344 [!] Text: 
----------------------------------
 Lorem ipsum dolor sit amet, consectetur adipiscing elit, sed do eiusmod 
-------------------------------------
producer2 -2061601024 [!] Text has been added! 

consumer1: -2069993728 [!] Text: 
----------------------------------
 tempor incididunt ut labore et dolore magna aliqua. Nulla facilisi nullam vehicula 
-------------------------------------
producer2 -2061601024 [!] Text has been added! 

producer2 -2061601024 [!] Text has been added! 

producer1 -2053208320 [!] Text has been added! 

producer2 -2061601024 [!] Text has been added! 

producer2 -2061601024 [!] Text has been added! 

consumer2: 2147481344 [!] Text: 
----------------------------------
 ipsum a arcu cursus vitae. Donec et odio pellentesque diam volutpat commodo. In pellentesque massa
-------------------------------------
consumer2: 2147481344 [!] Text: 
----------------------------------
 placerat duis ultricies lacus sed. Enim nec dui nunc mattis enim. Id consectetur purus
-------------------------------------
producer1 -2053208320 [!] Text has been added! 

consumer1: -2069993728 [!] Text: 
----------------------------------
 ut faucibus pulvinar elementum integer. Pharetra pharetra massa massa ultricies.
-------------------------------------
consumer2: 2147481344 [!] Text: 
----------------------------------
 Pretium quam vulputate dignissim suspendisse in est ante. Sapien nec sagittis
-------------------------------------
consumer1: -2069993728 [!] Text: 
----------------------------------
 aliquam malesuada bibendum. Quis hendrerit dolor magna eget est lorem ipsum. Amet cursus sit amet dictum sit.
-------------------------------------
producer2 -2061601024 [!] Text has been added! 

consumer1: -2069993728 [!] Text: 
----------------------------------
 Leo integer malesuada nunc vel risus commodo. Arcu non odio euismod lacinia. 
-------------------------------------
consumer1: -2069993728 [!] Text: 
----------------------------------
 Orci ac auctor augue mauris. Eu non diam phasellus vestibulum lorem. Nibh tortor id aliquet lectus proin nibh.
-------------------------------------
producer2 -2061601024 [!] Text has been added! 

consumer1: -2069993728 [!] Text: 
----------------------------------
 Proin fermentum leo vel orci porta non pulvinar neque. Molestie nunc non blandit massa enim nec dui nunc mattis. 
-------------------------------------
producer1 -2053208320 [!] Text has been added! 

consumer2: 2147481344 [!] Text: 
----------------------------------
 Habitant morbi tristique senectus et. Mattis rhoncus urna neque viverra justo nec. Ultricies integer quis auctor
-------------------------------------
producer1 -2053208320 [!] Text has been added! 

consumer1: -2069993728 [!] Text: 
----------------------------------
 elit sed vulputate mi sit. Justo eget magna fermentum iaculis.
-------------------------------------

\end{verbatim}


\newpage
\section{Продовження розминки.}
\large Теж саме, але не на семафорах, а на м’ютексі і
умовних змінних. 
C++

\subsection{Input file}

Такий самий (Див \ref{inputdata})

\subsection{Code}
\begin{lstlisting}[language=C++]
#include <stdlib.h>
#include <stdio.h>
#include <pthread.h>
#include <unistd.h>
#include <queue>
#include <semaphore.h>
#include <time.h>
using std::queue;

queue<char*> q;
FILE * file;
int done = 0;

pthread_mutex_t mutex;

void * produce(void * param)
{
	char * text;
	while(!done)
	{
		pthread_mutex_lock(&mutex);
		text = (char*)malloc(255 * sizeof(char));
		if (fgets(text, 256, file) != NULL)
		{
			q.push(text);
			printf("%s %d [!] Text has been added! \n\n", (char*)param, pthread_self());
		}
		else
		{
			done = 1;
		}
		pthread_mutex_unlock(&mutex);
		sleep(rand() % 4); // random sleep;
	}
	return NULL;

}

void * consume(void * param)
{
	while(!done || !q.empty()) // when both conditions is False exit the loop;
	{
		pthread_mutex_lock(&mutex);
		if(!q.empty())
		{
			printf("%s: %d [!] Text: \n-------------------\n %s--------------------\n",(char*)param , pthread_self(), q.front());
			q.pop();
		}
		pthread_mutex_unlock(&mutex);
		sleep(rand() % 4); // random sleep;
	}
	return NULL;
}

int main(int argv, char * argc[])
{
	srand(time(0));

	file = fopen("lorem_ipsum.txt", "r");

	pthread_t cons1, cons2;
	pthread_t prod1, prod2;
	pthread_mutex_init(&mutex, NULL);

	pthread_create(&prod1, NULL, produce, (void*)"producer1");
	pthread_create(&prod2, NULL, produce, (void*)"producer2");
	pthread_create(&cons1, NULL, consume, (void*)"consumer1");
	pthread_create(&cons2, NULL, consume, (void*)"consumer2");

	pthread_join(prod1, NULL);
	pthread_join(prod2, NULL);
	pthread_join(cons1, NULL);
	pthread_join(cons2, NULL);
	
	return 0;
}
\end{lstlisting}

\subsection{Program output}
\normalsize
\begin{verbatim}
producer1 -1394247936 [!] Text has been added! 

producer2 -1402640640 [!] Text has been added! 

producer2 -1402640640 [!] Text has been added! 

consumer1: -1411033344 [!] Text: 
-------------------
 Lorem ipsum dolor sit amet, consectetur adipiscing elit, sed do eiusmod 
--------------------
consumer2: -1419426048 [!] Text: 
-------------------
 tempor incididunt ut labore et dolore magna aliqua. Nulla facilisi nullam vehicula 
--------------------
consumer1: -1411033344 [!] Text: 
-------------------
 ipsum a arcu cursus vitae. Donec et odio pellentesque diam volutpat commodo. In pellentesque massa
--------------------
producer1 -1394247936 [!] Text has been added! 

producer2 -1402640640 [!] Text has been added! 

consumer1: -1411033344 [!] Text: 
-------------------
 placerat duis ultricies lacus sed. Enim nec dui nunc mattis enim. Id consectetur purus
--------------------
consumer2: -1419426048 [!] Text: 
-------------------
 ut faucibus pulvinar elementum integer. Pharetra pharetra massa massa ultricies.
--------------------
producer1 -1394247936 [!] Text has been added! 

producer2 -1402640640 [!] Text has been added! 

consumer1: -1411033344 [!] Text: 
-------------------
 Pretium quam vulputate dignissim suspendisse in est ante. Sapien nec sagittis
--------------------
consumer1: -1411033344 [!] Text: 
-------------------
 aliquam malesuada bibendum. Quis hendrerit dolor magna eget est lorem ipsum. Amet cursus sit amet dictum sit.
--------------------
producer1 -1394247936 [!] Text has been added! 

producer2 -1402640640 [!] Text has been added! 

consumer1: -1411033344 [!] Text: 
-------------------
 Leo integer malesuada nunc vel risus commodo. Arcu non odio euismod lacinia. 
--------------------
consumer1: -1411033344 [!] Text: 
-------------------
 Orci ac auctor augue mauris. Eu non diam phasellus vestibulum lorem. Nibh tortor id aliquet lectus proin nibh.
--------------------
producer2 -1402640640 [!] Text has been added! 

consumer1: -1411033344 [!] Text: 
-------------------
 Proin fermentum leo vel orci porta non pulvinar neque. Molestie nunc non blandit massa enim nec dui nunc mattis. 
--------------------
producer1 -1394247936 [!] Text has been added! 

consumer2: -1419426048 [!] Text: 
-------------------
 Habitant morbi tristique senectus et. Mattis rhoncus urna neque viverra justo nec. Ultricies integer quis auctor
--------------------
producer1 -1394247936 [!] Text has been added! 

consumer2: -1419426048 [!] Text: 
-------------------
 elit sed vulputate mi sit. Justo eget magna fermentum iaculis.
\end{verbatim}

\newpage
\section{Продовження розминки для тих, хто шукає пригод. Взаємне блокування}
Модифікуйте програму п. 1 так, щоби викликати взаємне блокування.
Для цього поміняйте місцями семафори. Переконайтесь у факті взаємного
блокування і отримайте задоволення. C++

\subsection{Input file}

Такий самий (Див \ref{inputdata})

\subsection{Code}
\begin{lstlisting}[language=C++]
#include <stdlib.h>
#include <stdio.h>
#include <pthread.h>
#include <unistd.h>
#include <queue>
#include <semaphore.h>
#include <time.h>
using std::queue;

queue<char*> q;
const int max_size = 5;
FILE * file;
int done = 0;

sem_t lock;
sem_t full;
sem_t empty;

void * produce(void * param)
{
	char * text;
	while(!done)
	{
		sem_wait(&lock);
		sem_wait(&empty);
		text = (char*)malloc(255 * sizeof(char));
		if (fgets(text, 256, file) != NULL)
		{
			q.push(text);
			printf("%s %d [!] Text has been added! \n\n", (char*)param, pthread_self());
		}
		else
		{
			done = 1;
		}
		sem_post(&lock);
		sem_post(&full);
		sleep(rand() % 4); // random sleep;
	}
	return NULL;

}

void * consume(void * param)
{
	while(!done || !q.empty()) // when both conditions is False exit the loop;
	{
		sem_wait(&lock);
		sem_wait(&full);
		if(!q.empty())
		{
			printf("%s: %d [!] Text: \n----------------------------------\n %s-------------------------------------\n",(char*)param , pthread_self(), q.front());
			q.pop();
		}
		sem_post(&lock);
		sem_post(&empty);
		sleep(rand() % 4); // random sleep;
	}
	return NULL;
}

int main(int argv, char * argc[])
{
	srand(time(0));

	sem_init(&lock, 0, 1);
	sem_init(&full, 0, 0);
	sem_init(&empty,0, max_size);
	file = fopen("lorem_ipsum.txt", "r");

	pthread_t cons1, cons2;
	pthread_t prod1, prod2;
	
	pthread_create(&prod1, NULL, produce, (void*)"producer1");
	pthread_create(&prod2, NULL, produce, (void*)"producer2");
	pthread_create(&cons1, NULL, consume, (void*)"consumer1");
	pthread_create(&cons2, NULL, consume, (void*)"consumer2");

	pthread_join(prod1, NULL);
	pthread_join(prod2, NULL);
	pthread_join(cons1, NULL);
	pthread_join(cons2, NULL);
	
	return 0;
}
\end{lstlisting}

\subsection{Program output}
\begin{verbatim}
producer1 457340672 [!] Text has been added! 

producer2 448947968 [!] Text has been added! 

consumer1: 335542016 [!] Text: 
----------------------------------
 Lorem ipsum dolor sit amet, consectetur adipiscing elit, sed do eiusmod 
-------------------------------------
consumer2: 440555264 [!] Text: 
----------------------------------
 tempor incididunt ut labore et dolore magna aliqua. Nulla facilisi nullam vehicula 
-------------------------------------

// Programm stucks

\end{verbatim}

\newpage

\section{Індивідуальне завдання. Варіант 1а.}
\textbf{\textit{ Обчислення числа $\pi$}}

\subsection{Code}

\begin{lstlisting}[language=C++]
#include <stdlib.h>
#include <stdio.h>
#include <unistd.h>
#include <pthread.h>

int step;

int get_sign(int num)
{
    if (num % 2 == 1)
        return -1;
    else
        return 1;
}

void * get_pi(void * param)
{
    double * sum = (double*)malloc(sizeof(double));
    *sum = 0;
    
    // 1.000.000.000 - the number of operations made by one thread; 
    for(int i = *(int*)param; i < 1000000000; i += step)
    {
        double val = (double)get_sign(i - 1)/(double)(2 * i - 1);
        *sum += val;
    }
    printf("Thread: %d | Sum: %.10f\n", *(int*)param, (*sum) * 4);
    return (void*)sum;
}

int main(int argv, char * argc[])
{
    if (argv != 2)
        return 1;

    const int threads_number = atoi(argc[1]);
    step = threads_number;
    printf("%d\n", threads_number);
    pthread_t threads[threads_number];

    for(int i = 0; i < threads_number; i++)
    {
        int * thread_number = (int*)malloc(sizeof(int));
        *thread_number = i+1;
        pthread_create(&(threads[i]), NULL, get_pi, (void*)(thread_number));
    }
    void * status;
    double sum;
    for(int i = 0; i < threads_number; i++)
    {
        pthread_join(threads[i], &status);
        sum += *((double*)status);
    }
    printf("%.30f\n", sum*4);
}
\end{lstlisting}
\newpage
\subsection{Program output}
\begin{enumerate}
	\normalsize
	\item Потоки: 4\vspace{-3mm}
	\begin{verbatim}
	./main_pi_number 4
	The number of threads: 4
	Thread: 4 | Sum: -10.0704942724
	Thread: 1 | Sum: 13.8627320688
	Thread: 3 | Sum: 10.3948401205
	Thread: 2 | Sum: -11.0454852622
	3.141592654591443434242137300316
	\end{verbatim}
	\item Потоків: 10\vspace{-3mm}
	\begin{verbatim}
	./main_pi_number 10
	The number of threads: 10
	Thread: 7 | Sum: 3.9582060413
	Thread: 1 | Sum: 7.7837051461
	Thread: 5 | Sum: 4.1308438777
	Thread: 4 | Sum: -4.2783503225
	Thread: 2 | Sum: -5.0883348170
	Thread: 10 | Sum: -3.8166580709
	Thread: 6 | Sum: -4.0313279990
	Thread: 3 | Sum: 4.5296268550
	Thread: 9 | Sum: 3.8551902692
	Thread: 8 | Sum: -3.9013083252
	3.141592654590677380355145942303
	\end{verbatim}
	\item Потоків: 21 \vspace{-3mm}
	\begin{verbatim}
	./main_pi_number 21
	The number of threads: 21
	Thread: 7 | Sum: 0.2597254755
	Thread: 8 | Sum: -0.2206794459
	Thread: 9 | Sum: 0.1911385221
	Thread: 6 | Sum: -0.3135240554
	Thread: 1 | Sum: 3.9358035006
	Thread: 21 | Sum: 0.0679289448
	Thread: 20 | Sum: -0.0720825669
	Thread: 4 | Sum: -0.5164410126
	Thread: 11 | Sum: 0.1495996492
	Thread: 18 | Sum: -0.0819575881
	Thread: 16 | Sum: -0.0946316452
	Thread: 10 | Sum: -0.1680698085
	Thread: 14 | Sum: -0.1114067237
	Thread: 13 | Sum: 0.1219713723
	Thread: 5 | Sum: 0.3920000009
	Thread: 3 | Sum: 0.7422297324
	Thread: 2 | Sum: -1.2725068838
	Thread: 12 | Sum: -0.1345090456
	Thread: 15 | Sum: 0.1023966007
	Thread: 19 | Sum: 0.0767292139
	Thread: 17 | Sum: 0.0878784180
	3.141592654588218014310996295535
	\end{verbatim}
	\item Потоків: 40\vspace{-3mm}
	\begin{verbatim}
	The number of threads: 40
	Thread: 5 | Sum: 1.3164628965
	Thread: 1 | Sum: 4.8795613042
	Thread: 3 | Sum: 1.6756619928
	Thread: 17 | Sum: 0.9753636539
	Thread: 12 | Sum: -1.0348055360
	Thread: 6 | Sum: -1.2339195141
	Thread: 15 | Sum: 0.9946777186
	Thread: 14 | Sum: -1.0062413923
	Thread: 16 | Sum: -0.9844655457
	Thread: 13 | Sum: 1.0194745976
	Thread: 18 | Sum: -0.9671857105
	Thread: 20 | Sum: -0.9530464350
	Thread: 2 | Sum: -2.2109109482
	Thread: 39 | Sum: 0.8837518345
	Thread: 10 | Sum: -1.0743718740
	Thread: 21 | Sum: 0.9468747594
	Thread: 24 | Sum: -0.9310626161
	Thread: 7 | Sum: 1.1762933849
	Thread: 19 | Sum: 0.9597854539
	Thread: 8 | Sum: -1.1336359177
	Thread: 29 | Sum: 0.9109739392
	Thread: 33 | Sum: 0.8985496355
	Thread: 30 | Sum: -0.9076220788
	Thread: 31 | Sum: 0.9044440858
	Thread: 37 | Sum: 0.8882764306
	Thread: 22 | Sum: -0.9411938543
	Thread: 40 | Sum: -0.8816176832
	Thread: 38 | Sum: -0.8859693537
	Thread: 25 | Sum: 0.9265158102
	Thread: 28 | Sum: -0.9145173433
	Thread: 4 | Sum: -1.4452387943
	Thread: 11 | Sum: 1.0528249967
	Thread: 9 | Sum: 1.1006790416
	Thread: 36 | Sum: -0.8906799226
	Thread: 26 | Sum: -0.9222630166
	Thread: 35 | Sum: 0.8931874523
	Thread: 32 | Sum: -0.9014244786
	Thread: 23 | Sum: 0.9359406291
	Thread: 27 | Sum: 0.9182725719
	Thread: 34 | Sum: -0.8958075197
	3.141592654589066224701809915132
	\end{verbatim}
\end{enumerate}
\newpage

\section{Індивідуальне завдання. Варіант 1б.}
\textbf{\textit{Обчислення $\pi$ аж поки не набридне}}
\subsection{Code}

\begin{lstlisting}[language=C++]
#include <stdlib.h>
#include <stdio.h>
#include <unistd.h>
#include <pthread.h>
#include <signal.h>

int step;
int done = 0;
pthread_mutex_t mutex;


int get_sign(int num)
{
    if (num % 2 == 1)
        return -1;
    else
        return 1;
}


int next_check = 0;

void * get_pi(void * param)
{
    double * sum = (double*)malloc(sizeof(double));
    *sum = 0;
    int i = *(int*)param;
    
    while(1)
    {
        pthread_mutex_lock(&mutex);
        next_check += 10000;
        pthread_mutex_unlock(&mutex);
        for(; i < next_check; i += step)
        {
            double val = (double)get_sign(i - 1)/(double)(2 * i - 1);
            *sum += val;
        }
        if(!done)
            continue;
        else
        {    
            printf("Thread: %d | Sum: %.10f\n", *(int*)param, (*sum) * 4);
            pthread_exit((void*)sum);
        }   
    }
    
}

void sigint_handler(int sig)
{
    printf("\nGot Signal: %d\n", sig);
    if (sig == 2)
    {
        done = 1;
    }
}

int main(int argv, char * argc[])
{
    if (argv != 2)
        return 1;

    signal(SIGINT, sigint_handler);
    const int threads_number = atoi(argc[1]);
    step = threads_number;
    pthread_t threads[threads_number];
    pthread_mutex_init(&mutex, NULL);
    printf("The number of threads: %d\n", threads_number);

    for(int i = 0; i < threads_number; i++)
    {
        int * thread_number = (int*)malloc(sizeof(int));
        *thread_number = i+1;
        pthread_create(&(threads[i]), NULL, get_pi, (void*)(thread_number));
    }
    void * status;
    double sum;
    for(int i = 0; i < threads_number; i++)
    {
        pthread_join(threads[i], &status);
        sum += *((double*)status);
    }
    printf("Pi number: %.25f\n", sum*4);
}
\end{lstlisting}
\subsection{Program output}
\begin{enumerate}
	\normalsize
	\item Приблизно 1 секунда виконання.\vspace{-3mm}
	\begin{verbatim}
		./main_pi_number_signal 4
		The number of threads: 4
		^C
		Got Signal: 2
		Thread: 2 | Sum: -11.0214799434
		Thread: 1 | Sum: 13.8387267501
		Thread: 3 | Sum: 10.3708348016
		Thread: 4 | Sum: -10.0464889576
		Pi number: 3.1415926506793532269057323
	\end{verbatim}
	\item Приблизно 2 секунда виконання.\vspace{-3mm}
	\begin{verbatim}
		./main_pi_number_signal 4
		The number of threads: 4
		^C
		Got Signal: 2
		Thread: 4 | Sum: -9.9630677656
		Thread: 3 | Sum: 10.2874136140
		Thread: 1 | Sum: 13.7553055621
		Thread: 2 | Sum: -10.9380587557
		Pi number: 3.1415926548321806421881774
	\end{verbatim}
	\item Приблизно 7 секунд виконання\vspace{-3mm}
	\begin{verbatim}
		./main_pi_number_signal 4
		The number of threads: 4
		^C
		Got Signal: 2
		Thread: 2 | Sum: -10.7204736458
		Thread: 4 | Sum: -9.7454826608
		Thread: 1 | Sum: 13.5377204529
		Thread: 3 | Sum: 10.0698285035
		Pi number: 3.1415926498115744891492795
	\end{verbatim} 
	\item Приблизно 5 секунд виконання з 10 потоками\vspace{-3mm}
	\begin{verbatim}
	./main_pi_number_signal 10
	The number of threads: 10
	^C
	Got Signal: 2
	Thread: 5 | Sum: 3.9751028638
	Thread: 9 | Sum: 3.6994492563
	Thread: 3 | Sum: 4.3738858406
	Thread: 2 | Sum: -4.9325938024
	Thread: 6 | Sum: -3.8755869854
	Thread: 1 | Sum: 7.6279641313
	Thread: 8 | Sum: -3.7455673121
	Thread: 4 | Sum: -4.1226093083
	Thread: 7 | Sum: 3.8024650278
	Thread: 10 | Sum: -3.6609170558
	Pi number: 3.1415926557688838016701993
	\end{verbatim} 
	\item Приблизно 7 секунд виконання з 20 потоками\vspace{-3mm}
	\begin{verbatim}
	./main_pi_number_signal 20
	The number of threads: 20
	^C
	Got Signal: 2
	Thread: 11 | Sum: 1.2634154114
	Thread: 1 | Sum: 5.2439507117
	Thread: 19 | Sum: 0.1398662615
	Thread: 7 | Sum: 0.5243703295
	Thread: 5 | Sum: 1.9629374717
	Thread: 18 | Sum: -1.6508021123
	Thread: 8 | Sum: -0.3686608871
	Thread: 20 | Sum: -1.2857671313
	Thread: 4 | Sum: -0.9717487678
	Thread: 12 | Sum: -1.6687581321
	Thread: 9 | Sum: 1.8296835744
	Thread: 17 | Sum: 1.5143951665
	Thread: 15 | Sum: 1.3400078687
	Thread: 10 | Sum: -1.7789068754
	Thread: 14 | Sum: -1.7278023938
	Thread: 6 | Sum: -1.9663967789
	Thread: 2 | Sum: -1.5137887941
	Thread: 13 | Sum: 1.7425065355
	Thread: 3 | Sum: 2.2159906903
	Thread: 16 | Sum: -1.6100748561
	Pi number: 3.1415926553421256271581115
	\end{verbatim} 
	
	Через обмеження double, якщо виконувати довше 10 секунд, процессор встигає зробити стільки операцій, що не вистачає розміру double, і дані починають плисти та число $\pi$ стає відрізнятися. Щоб вирішити цю проблему, треба використати бібліотеку на java, що дає змогу працювати з великою кількістю знаків після точки.
	
	Також на комп'ютері де я виконую лабораторну роботу встановлений достатньо швидкий процессор Intel Core i7 -3600MQ, тому обчислення відбуваються досить швидко. Меше ніж за секунду мені показує такий результат, \verb|Pi number: 3.1415926628848360735446477|, що є доволі точним результатом.
	
	\item Якщо зачекати 20 секунд можна отримати такий результат\vspace{-3mm}
	\begin{verbatim}
		./main_pi_number_signal 4
		The number of threads: 4
		^C
		Got Signal: 2
		Thread: 2 | Sum: -2.4247871401
		Thread: 3 | Sum: 6.5764089532
		Thread: 4 | Sum: -6.5183746496
		Thread: 1 | Sum: 11.2669042742
		Pi number: 8.9001514377313313275408291
	\end{verbatim}
		
\end{enumerate}

\section{Висновки}
\large
Впродовж виконання цієї лабораторної роботи я дізнався: як створювати багатопотокові программи, блокувати потоки, коли це потрібно за допомогою семафорів, мютексів. Розглянув такий варіант семафору як locker. Навчився надсилати до программи сигнал SIGINT та обробляти його так, як треба розробнику, без термінового завершення программи.

\end{document}