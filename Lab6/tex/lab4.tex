\documentclass{article}

\usepackage{fancyhdr}
\usepackage{cmap}
\usepackage[T2A]{fontenc}
\usepackage[utf8]{inputenc}
\usepackage[english,russian,ukrainian]{babel}
\usepackage{indentfirst}
\usepackage{graphicx}
\usepackage{hyperref}
\usepackage{titlesec}
\usepackage{fancyhdr}
\usepackage{geometry}

\hypersetup{
	colorlinks,
	citecolor=black,
	linkcolor=black,
	filecolor=black,
	urlcolor=black
}

\geometry{
	a4paper,
	total={165mm,247mm},
	left=20mm,
	top=30mm,
}

\pagestyle{fancy}
\fancyhf{}
\renewcommand{\headrulewidth}{0.5pt}
%\renewcommand{\footrulewidth}{0.1pt}
\rhead{\LaTeX}
\lhead{Вячеслав Козачок}

\cfoot{\thepage}
%\titleformat{\section}
%{\normalfont\Large\bfseries}{\thesection}{1em}{}

%\titleformat*{\subsection}{\large\bfseries}

 \usepackage{listings}
 \usepackage{xcolor}
 
 \definecolor{codegreen}{rgb}{0,0.6,0}
 \definecolor{codegray}{rgb}{0.5,0.5,0.5}
 \definecolor{codepurple}{rgb}{0.58,0,0.82}
 \definecolor{backcolour}{rgb}{0.99,0.99,0.99}
 \definecolor{codeblue}{rgb}{0.1,0.1,0.99}
 
 \lstdefinestyle{mystyle}{
 	backgroundcolor=\color{backcolour},   
 	commentstyle=\color{codegreen},
 	keywordstyle=\color{codeblue},
 	numberstyle=\tiny\color{codegray},
 	stringstyle=\color{codepurple},
 	basicstyle=\ttfamily\footnotesize,
 	breakatwhitespace=false,         
 	breaklines=true,                 
 	captionpos=b,                    
 	keepspaces=true,                 
 	numbers=left,                    
 	numbersep=5pt,                  
 	showspaces=false,                
 	showstringspaces=false,
 	showtabs=false,                  
 	tabsize=4
 }
 
 \lstset{style=mystyle}
 
 
\begin{document}
 \begin{titlepage}
	
	\begin{frame}[t]
		\raisebox{-10mm}[10pt][0pt]{%
			\makebox[\textwidth][c]{\includegraphics[width=\textwidth]{institute.jpg}}
		}
	\end{frame}\\
	\large
	\centering{
		\vspace{5mm}\\
		Міністерство освіти і науки України \\
		Національний технічний університет України \\
		"Київський політехнічний інститут імені Ігоря Сікорського"\\
		Фізико-технічний інститут
	}
	\vspace{3cm}\\
	\centering{
		{\Huge\textbf{Операційні системи} \\ \vspace{0.15cm}}
		{\huge Лабораторна №4}
	}
	
	\vspace{8cm}
	\Large
	\begin{flushright}
		Виконав:\\
		Студент групи ФБ-82\\
		\textbf{Козачок Вячеслав}\\
		Перевірив: \\
		Кіреєнко О.В.
	\end{flushright}
	\vfill
	
	\centering{Київ - 2020}
	
\end{titlepage}
\large

\section{Завдання до виконання}
\begin{enumerate}
	\item Для початку можна взяти демонстраційну програму, запропоновану.
	\begin{lstlisting}[language=C]
#include <iostream>
#include <string>
// Required by for routine
#include <sys/types.h>
#include <unistd.h>
#include <stdlib.h>
// Declaration for exit()
using namespace std;
int globalVariable = 2;
int main()
{
	string sIdentifier;
	int iStackVariable = 20;
	pid_t pID = fork();
	if (pID == 0)
	// child
	{
		// Code only executed by child process
		sIdentifier = "Child Process: ";
		globalVariable++;
		iStackVariable++;
	}
	else if (pID < 0)
	// failed to fork
	{
		cerr << "Failed to fork" << endl;
		exit(1);
		// Throw exception
	}
	else
	// parent
	{
		// Code only executed by parent process
		sIdentifier = "Parent Process:";
	}
	// Code executed by both parent and child.
	cout << sIdentifier;
	cout << " Global variable: " << globalVariable;
	cout << " Stack variable: " << iStackVariable << endl;
}
\end{lstlisting}
	\item Скомпілюйте програму. (Вважаємо,текст збережено у файлі myforktest.cpp) \\
	\texttt{g++ -o myforktest myforktest.cpp} \\
	Увага! Пам’ятайте, що не можна називати власні програми просто
	test! test – це вбудована команда shell (з якою ви вже зустрічалися
	в Роботі №4).
	\item Запустіть програму myforktest. У якій послідовності виконуються
	батьківський процес і процес-нащадок? Чи завжди цей порядок
	дотримується?
	\item Додайте затримку у виконання одного або обох з цих процесів
	(функція sleep(), аргумент — затримка у секундах). Чи змінились
	результати виконання?
	\item Додайте цикл, який забезпечить кількаразове повторення дій після
	виклику fork(). Які результати показують процеси (значення
	глобальної змінної і змінної, що визначена у стеку)? Поясніть.
	\item Спробуйте у первинній програмі (без циклу) замість виклику fork()
	здійснити виклик vfork(). У чому різниця роботи цих двох викликів?
	Чи виникає помилка (якщо так, то яка)? У чому причина? Як "змусити”
	працювати виклик vfork()? Які результати тепер показують процеси
	(значення глобальної змінної і змінної, що визначена у стеку)?
	Поясніть.
	\item Тепер додайте виклик exec() у код процесу-нащадка. Для початку
	використайте простішу функцію execl(). Варіант виклику на
	прикладі утиліти ls:
	execl(або"/bin/ls", "/bin/ls", "-a", "-l", (абоchar *) 0);
	У наведеному прикладі передаються аргументи командного рядка.
	\item Проведіть експерименти з викликом різних програм, у тому числі ps,
	bash, а також з викликами execl() у батьківському процесі. Як
	запустити фоновий процес-нащадок? Як процес-нащадок дізнається
	власний PID? PID батьківського процесу?
	\item Усі отримані результати і відповіді на запитання, які були задані вище,
	іть у вигляді протоколу.
\end{enumerate}
\newpage

\section*{Виконання роботи}
\subsection*{Task 2}
\begin{lstlisting}[language=C]
#include <iostream>
#include <string>
// Required by for routine
#include <sys/types.h>
#include <unistd.h>
#include <stdlib.h>
// Declaration for exit()
using namespace std;

int globalVariable = 2;

int main()
{
	string sIdentifier;
	int iStackVariable = 20;
	pid_t pID;
	pID = fork();
	if (pID == 0)
	{
		// Code only executed by child process
		sIdentifier = "Child Process: ";
		globalVariable++;
		iStackVariable++;
	}
	else if (pID < 0)
	{
		cerr << "Failed to fork" << endl;
		exit(1);
		// Throw exception
	}
	else
	{
		// Code only executed by parent process
		sIdentifier = "Parent Process:";
		sleep(1);
	}
	// Code executed by both parent and child.
	cout << sIdentifier;
	cout << " Global variable: " << globalVariable;
	cout << " Stack variable: " << iStackVariable 
		<< " Pid: " << pID << endl;
}
\end{lstlisting}
Terminal:
\begin{lstlisting}[language=BASH]
g++ -g main.cpp -o main
./main
Child Process:  Global variable: 3 Stack variable: 21 Pid: 0
Parent Process: Global variable: 2 Stack variable: 20 Pid: 20372
\end{lstlisting}

\subsection*{Task 3}
Послідовність не змінюється. Через деяку кількість викликів не змінилась послідовність. Спочатку відпрацьовує батьківський процесс, а вже потім дочірній.
\subsection*{Task 4}
\begin{lstlisting}[language=C]
...
else
{
// Code only executed by parent process
sIdentifier = "Parent Process:";
sleep(1);
}
...
\end{lstlisting}
Terminal:
\begin{lstlisting}[language=BASH]
g++ -g main.cpp -o main
./main
Child Process:  Global variable: 3 Stack variable: 21 Pid: 0
Parent Process: Global variable: 2 Stack variable: 20 Pid: 20625
\end{lstlisting}

Якщо додати \texttt{sleep} в батьківську частину коду, то спочатку відпрацьовує дочірня частина, а потім батьківська, проте, якщо додати в дочірню частину, то спочатку виводить батьківську частину, дозволяє вводити команди в \texttt{shell} проте программа дитячого процесу продовжує вивід у оболочку.

\begin{lstlisting}[language=BASH]
g++ -g main.cpp -o main
./main
Parent Process: Global variable: 2 Stack variable: 20 Pid: 21091
[eski@eski-pc Lab6]$ Child Process:  Global variable: 3 Stack variable: 21 Pid: 0
\end{lstlisting}


\subsection*{Task 5}
\begin{lstlisting}[language=C]
int main()
{
	string sIdentifier;
	int iStackVariable = 20;
	pid_t pID;
	pID = fork();
	for(int i = 0; i < 4; i++)
	{
		if (pID == 0)
		{
			// Code only executed by child process
			sIdentifier = "Child Process: ";
			globalVariable++;
			iStackVariable++;
		}
		else if (pID < 0)
		{
			cerr << "Failed to fork" << endl;
			exit(1);
			// Throw exception
		}
		else
		{
			// Code only executed by parent process
			sIdentifier = "Parent Process:";
			sleep(1);
		}
		// Code executed by both parent and child.
		cout << sIdentifier;
		cout << " Global variable: " << globalVariable;
		cout << " Stack variable: " << iStackVariable 
			<< " Pid: " << pID << endl;
	}
}
\end{lstlisting}
Terminal:
\begin{lstlisting}[language=BASH]
g++ -g main.cpp -o main
./main
Child Process:  Global variable: 3 Stack variable: 21 Pid: 0
Child Process:  Global variable: 4 Stack variable: 22 Pid: 0
Child Process:  Global variable: 5 Stack variable: 23 Pid: 0
Child Process:  Global variable: 6 Stack variable: 24 Pid: 0
Parent Process: Global variable: 2 Stack variable: 20 Pid: 20848
Parent Process: Global variable: 2 Stack variable: 20 Pid: 20848
Parent Process: Global variable: 2 Stack variable: 20 Pid: 20848
Parent Process: Global variable: 2 Stack variable: 20 Pid: 20848
\end{lstlisting}

Якщо поставити sleep(1) в частину батьківську, то кожна стрічка ``Parent process'' виводиться через секунду після попередньої.
\newpage 
\subsection*{Task 6}
\begin{lstlisting}[language=C]
...
pid_t pID;
pID = vfork();
if (pID == 0)
{
...
\end{lstlisting}
Terminal:
\begin{lstlisting}[language=BASH]
g++ -g main.cpp -o main
./main
Child Process:  Global variable: 3 Stack variable: 21 Pid: 0
Parent Process: Global variable: 3 Stack variable: 21 Pid: 21255
*** stack smashing detected ***: terminated
make: *** [Makefile:3: c] Aborted (core dumped)
\end{lstlisting}

\texttt{vfork()} не створює копію батьківського процесу, а створює поділюваний з батьківським процессом адресний простір до тих пір, поки не буде викликана функція \texttt{\_exit} чи одна з функцій сімейства \texttt{exec}
\\

Щоб не ставалося помилки, треба закрити дочірній процес викликаний викликом\texttt{vfork()}.
\\

Але оскільки ми отримуємо адресний простір батьківського процесу, то дочірній процес може змінювати змінні батьківського, отже глобальна змінна, як і зміна стеку, зміняться.

\begin{lstlisting}[language=BASH]
...
 if (pID == 0)
{
	// Code only executed by child process
	sIdentifier = "Child Process: ";
	sleep(1);
	globalVariable++;
	iStackVariable++;
	exit(EXIT_SUCCESS);
}
...
\end{lstlisting}
Terminal:
\begin{lstlisting}[language=BASH]
g++ -g main.cpp -o main
./main
Parent Process: Global variable: 3 Stack variable: 21 Pid: 23049
\end{lstlisting}

Проте  як ми можемо бачити, то процес дочірній завершився та не вивів тексту у стандартний потік.
\\

\newpage
\subsection*{Task 7}
\begin{lstlisting}[language=C]
 if (pID == 0)
{
	// Code only executed by child process
	sIdentifier = "Child Process: ";
	sleep(1);
	globalVariable++;
	iStackVariable++;
	execl("/bin/ls", "-a", "-l", 0);
	_exit(EXIT_SUCCESS);
}
\end{lstlisting}
Terminal:
\begin{lstlisting}[language=BASH]
[eski@eski-pc Lab6]$ make
g++ -g main.cpp -o main
./main
Parent Process: Global variable: 2 Stack variable: 20 Pid: 24651
[eski@eski-pc Lab6]$ total 216
drwxr-xr-x 3 eski eski  4096 Mar 19 22:05 .
drwxr-xr-x 8 eski eski  4096 Mar 19 17:42 ..
-rw-r--r-- 1 eski eski   879 Mar 19 22:02 lab4.aux
-rw-r--r-- 1 eski eski 28894 Mar 19 22:02 lab4.log
-rw-r--r-- 1 eski eski   211 Mar 19 22:02 lab4.out
-rwxr-xr-x 1 eski eski 74960 Mar 19 22:05 main
-rw-r--r-- 1 eski eski  1025 Mar 19 22:04 main.cpp
-rwxr-xr-x 1 eski eski 74744 Mar 19 19:40 main.o
-rw-r--r-- 1 eski eski    46 Mar 19 19:45 Makefile
drwxr-xr-x 2 eski eski  4096 Mar 19 22:04 tex
-rw-r--r-- 1 eski eski   626 Mar 19 22:00 title.aux
\end{lstlisting}

Використання ps:

\begin{lstlisting}[language=C]
if (pID == 0)
{
	// Code only executed by child process
	sIdentifier = "Child Process: ";
	sleep(1);
	globalVariable++;
	iStackVariable++;
 -> execl("/bin/ps", "/bin/ps", "-U", "eski", 0);
	_exit(EXIT_SUCCESS);
}
\end{lstlisting}

\begin{lstlisting}[language=BASH]
[eski@eski-pc Lab6]$ make
g++ -g main.cpp -o main
./main
Parent Process: Global variable: 2 Stack variable: 20 Pid: 25397
[eski@eski-pc Lab6]$     PID TTY          TIME CMD
1506 ?        00:00:00 systemd
1507 ?        00:00:00 (sd-pam)
1518 ?        00:00:04 kwalletd5
1519 ?        00:00:00 startplasma-x11
1529 ?        00:00:19 dbus-daemon
1543 ?        00:00:00 start_kdeinit
1544 ?        00:00:00 kdeinit5
1560 ?        00:00:05 klauncher
\end{lstlisting}

\subsection*{Task 8}
\begin{itemize}
	\item Ми можемо запустити фоновий процес-нащадок
	використовуючи: \texttt{waitpid()}
	
	\item \texttt{fork()} повертає дочірній PID батьківському процессу.
	
	\item Процес-нащадок може знайти PID батьківському процессу
	використовуючи \texttt{getppid()}.
	
\end{itemize}
\newpage
\section*{Висновки}
В цій роботі я зрозумів, як у ОС UNIX здійснюється створення дочірніх процесів батьківським процесом. Зрозумів, коли процеси користуються спільним адресним простором, а коли для кожного з них виділяється свій адресний простір.
\end{document}

